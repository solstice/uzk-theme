%% UzK - A BEAMER THEME FOR THE UNIVERSITY OF COLOGNE
%% http://solstice.github.com/uzk-theme/

\documentclass{beamer}
\usepackage[ngerman]{babel}
\usepackage[latin1]{inputenc}
\usepackage[T1]{fontenc}

%% Falls Anzeige der \sections, \subsections etc. gewuenscht, kann zB.
%% das infolines theme geladen werden. Wichtig ist jedoch, dass andere
%% Themes _vor_ dem UzK-Theme geladen werden.
%\useoutertheme{infolines}

%% Falls keine der Optionen zur Bestimmung der Fusszeile uebergeben werden    %%
%% werden alle Fakultaetsfarben verwendet. ---------------------------------- %%
\usetheme[%
%wiso,        %% Wiso-Fakultaet
%jura,        %% Rechtswissenschaftliche Fakultaet
%medizin,     %% Medizinische Fakultaet
%philo,       %% Philosophische Fakultaet
%matnat,      %% Mathematisch-Naturwissenschaftliche Fakultaet
%human,       %% Humanwissenschaftliche Fakultaet
%verw,        %% Universitaetsverwaltung
%nav,         %% Schaltet die Navigationssymbole ein
%latexfonts,  %% Verwendet die latex-beamer-Standardschrift
%colorful,    %% Farbige Balken im infolines-Theme
%squares,     %% Aufzaehlungspunkte rechteckig
%nologo,      %% Kein Logo im Seitenhintergrund
]{UzK}

\title{Beamer-Theme \texttt{UzK} f�r Pr�sentationen im Corporate
  Design der Universit�t zu K�ln}

\author[David Kusterer \and Bernd Wei�]%
{David Kusterer\thanks{ \href{mailto:kusterer@uni-koeln.de}{kusterer@uni-koeln.de} }%
  \and%
  Bernd Wei�\thanks{\href{mailto:bernd.weiss@wiso.uni-koeln.de}{bernd.weiss@wiso.uni-koeln.de}}}

\institute[Forschungsinstitut f�r Soziologie]{%
Forschungsinstitut f�r Soziologie \\
Greinstra�e 2\\
50939 K�ln}

\begin{document}

\begin{frame}[titlepage]
  \titlepage
\end{frame}

\begin{frame}
  \frametitle{Allgemeines}

  \begin{itemize}
  \item Mit diesem \emph{beamer theme} ist es m�glich, Pr�sentationen in
    \LaTeX{} mit der Beamer-Klasse zu erstellen, die dem Corporate Design der
    Universit�t zu K�ln entsprechen
  \item Auf die Beamer-Klasse wird in diesem Dokument nicht n�her eingegangen,
    n�here Informationen finden Sie unter
    \url{http://latex-beamer.sourceforge.net/}
  \end{itemize}

\end{frame}

\begin{frame}[fragile]
  \frametitle{Laden des Themes}
  \begin{block}{Das Theme kann mit den folgenden Optionen geladen werden}
    \begin{small}
\begin{verbatim}
\usetheme[%
% uk,      %% Farben aller Fakultaeten
wiso,     %% Wiso-Fakultaet
% jura,    %% Rechtswissenschaftliche Fakultaet
% medizin, %% Medizinische Fakultaet
% philo,   %% Philosophische Fakultaet
% matnat,  %% Mathematisch-Naturwissenschaftliche Fakultaet
% human,   %% Humanwissenschaftliche Fakultaet
% verw,    %% Universitaetsverwaltung
]{UzK}
\end{verbatim}
    \end{small}

  \end{block}
\end{frame}

\begin{frame}
  \frametitle{Die Fu�zeile}

  \begin{itemize}
  \item Es stehen verschiedene Fu�zeilen zur Auswahl, die als Option
    beim Laden des \emph{themes} �bergeben werden:
    \begin{itemize}
    \item Balken mit allen Fakult�tsfarben (Option \texttt{uk})
    \item Balken in jeweils einer Fakult�tsfarbe (Optionen \texttt{wiso, jura,
        medizin, philo, matnat, human, verw})\footnote{Es werden die offiziellen
        RGB-Werte aus dem 2-D Handbuch Corporate Design verwendet.}
    \end{itemize}
  \item "`Universit�t zu K�ln"' sowie der Name der Fakult�t sind im
    Theme definiert, das Institut oder Seminar kann mit dem Befehl
    \texttt{\textbackslash institute\{\}} festgelegt werden
  \item Die Optionen sind im Quellcode dieser Pr�sentation dokumentiert
  \end{itemize}

\end{frame}

\begin{frame}
  \frametitle{Englische Pr�sentationen}
  \begin{itemize}
  \item Der Universit�ts- sowie die Fakult�tsnamen werden
    standardm��ig auf Deutsch angezeigt.
  \item �bergeben Sie dem Paket \texttt{babel} die Option
    \texttt{english}, so werden diese Namen entsprechen angepasst.
  \item Die �bersetzungen k�nnen in der Theme-Datei
    \texttt{beamerthemeUzK.sty} ge�ndert werden
  \end{itemize}

\end{frame}

\begin{frame}
  \frametitle{\texttt{block}-Umgebungen}
  \begin{block}{Standard (\texttt{block})}
    Verwendet die Farbe "`Blaugrau Mittel"' als Blocktitel-Hintergrund
  \end{block}

  \begin{exampleblock}{\texttt{exampleblock}}
    Bei Verwendung der Fu�zeile mit allen Fakult�tsfarben
    Titelhintergrund in Wiso-Gr�n, sonst in der jeweiligen
    Fakult�tsfarbe
  \end{exampleblock}

  \begin{alertblock}{\texttt{alertblock}}
    Verwendet das Rot der Folientitel
  \end{alertblock}

\end{frame}


\begin{frame}
  \frametitle{Installation}
  \begin{itemize}
  \item Das Theme besteht aus den Dateien
    \texttt{beamerthemeUzK.sty} und \texttt{beamercolorthemeUzK.sty}
    sowie den Grafikdateien \texttt{logo.pdf} und
    \texttt{logo-small.pdf}.
  \item Das Theme kann auf zwei Arten verwendet werden:
    \begin{enumerate}
    \item Die vier Dateien werden in den selben Ordner wie die zu
      erstellende Pr�sentation gelegt
    \item Die vier Dateien werden im lokalen \emph{texmf}-Baum abgelegt
    \end{enumerate}
  \item Die zweite Variante ist der ersten vorzuziehen, da das Theme
    so an einem zentralen Ort vorliegt
  \end{itemize}
\end{frame}


\begin{frame}
  \frametitle{ToDo}

  \begin{block}{Was noch zu tun ist\ldots}
    \begin{itemize}
    \item Erstellen einer eigenen Titelseite
    \item \ldots
    \end{itemize}
  \end{block}

\end{frame}

\end{document}
